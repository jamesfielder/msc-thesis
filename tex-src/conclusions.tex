In this project, we have primarily strived to understand the underlying physics which affects the flight
of a golf ball. In doing so we have been lead to consider the drag and lift functions in depth, as
these functions contain much of the subtlety that makes golf into such a popular game for people all
over the world.

The physical basis on which we have set this project is quite well established: as far back as
\citet{Bearman1976} equations of motion similar to the ones given in this project have been known of
and used successfully to calculate golf ball trajectories. The extensions to different forms of 
$c_D$ and $c_L$ are fairly new, and more research will be needed to ascertain which if any of these
forms are the most appropriate for the modelling of golf ball flight.

The use of least squares to estimate the parameters in models of golf ball flight seems to be fairly
novel. Unfortunately, performing these estimations seems to present a number of challenges which
we have been unable to overcome in this work. In the future, with better understanding of the parameter
space and how the parameters affect the resultant trajectories, these problems could well be solved.

\section{Possible Future Work}

There are a number of potential improvements to this project which could be investigated
\begin{enumerate}
\item In this project we have completely neglected to model spin reduction during the course
of the flight of the ball. Investigating this effect could make a difference to the resultant 
trajectories.
\item A more comprehensive understanding how the parameters in either the \citet{Lieberman2001} form 
for $c_D$ and $c_L$ or the $\tanh$ form change the trajectory as they differ would help to constrain
the least squares problem and make it easier to solve.
\item Applying more advanced inverse problems techniques to the least squares parameter estimation
problem could also help to improve this estimation.
\item Using more experimental data and statistical inverse problems techniques such as Monte Carlo 
estimation to obtain a more probabilistic driven estimation for the model parameters. 
\end{enumerate}

Overall however, we feel that this project has advanced the understanding of how one might use 
experimental data within the modelling of golf ball flight at least a little and the avenues explored
within will hopefully present more results as more research is conducted in this area.