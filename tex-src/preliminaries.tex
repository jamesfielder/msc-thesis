In order to devise a simple model for golf ball flight we first must understand some
prerequisite physics for projectiles and fluid dynamics for the airflow over the ball. Understanding
how the fluid flows over the surface of the ball is crucial to understanding the difference between
the flight of a golf ball and that of a standard projectile. Quantifying this effect will be a large
component of this project.

There has been significant work done previously in understanding the fluid dynamics around a golf ball
and how a golf ball flies. We will attempt to review some of this literature in this chapter and summarise
previous work on the topic.

First though, we must understand how normal projectiles fly without taking into account fluid dynamics effects.
\section{Projectile Motion}
A projectile is a body fired into the air by an initial impulse and then allowed to fall back to ground under the
action of gravity alone. This is the most naive and simplistic model of golf ball flight, completely
ignoring all aerodynamic effects, however we must understand it before building up to a more
complex model.

Consider motion in a 2 dimensional plane, labelled by $x$ along the horizontal and $y$ along the vertical
A projectile is given an initial speed of the form $\vv_0 = (v_x, v_y)$ at an angle $\alpha$ to
the horizontal. We set the origin of the coordinate system to be the point at the start of the
trajectory, $(x_0, y_0) = (0,0)$. In this problem the acceleration on the projectile, after the initial
impulse, is constant and of the form
\begin{equation}
a_x = 0, \quad a_y = -g
\end{equation}
where $g$ is the acceleration due to gravity. Since the acceleration is constant we can use
the standard formulas for motion under constant acceleration to derive the dynamics of the
projectile \citet{yandf}
\section{Basic Aerodynamics}\begin{equation}
\rho \matdef{\vec{q}} = - \nabla p + \mu \nabla^2 \vec{q} + f
\end{equation}

\section{Boundary Layers}

