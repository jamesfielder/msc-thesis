In order to devise a simple model for golf ball flight we first must understand some
prerequisite physics for projectiles and fluid dynamics for the airflow over the ball. Understanding
how the fluid flows over the surface of the ball is crucial to understanding the difference between
the flight of a golf ball and that of a standard projectile. Quantifying this effect will be a large
component of this project.

There has been significant work done previously in understanding the fluid dynamics around a golf ball
and how a golf ball flies. We will attempt to review some of this literature in this chapter and summarise
previous work on the topic.

First though, we must understand how normal projectiles fly without taking into account fluid dynamics effects.
\section{Projectile Motion}
A projectile is a body fired into the air by an initial impulse and then allowed to fall back to ground under the
action of gravity alone. This is the most naive and simplistic model of golf ball flight, completely
ignoring all aerodynamic effects, however we must understand it before building up to a more
complex model.

Consider motion in a 2 dimensional plane, labelled by $x$ along the horizontal and $y$ along the vertical.
A projectile is given an initial speed of the form $\vv_0 = (v_x, v_y)$. We set the origin of the coordinate system to be the point at the start of the
trajectory, $(x_0, y_0) = (0,0)$. In this problem the acceleration on the projectile, after the initial
impulse, is constant and of the form
\begin{equation} \label{grav}
a_x = 0, \quad a_y = -g
\end{equation}
where $g$ is the acceleration due to gravity. Since the acceleration is constant we can use
the standard formulas for motion under constant acceleration to derive the dynamics of the
projectile \citet{yandf}, which are
\begin{subequations} \label{suvat}
\begin{align}
v &= v_0 + at \label{v} \\
x &= v_0 t + \frac{1}{2} a t^2 \label{xsuvat} \\
v^2 &= v_0^2 + 2ax \\
x &= \left(\frac{v_0 + v}{2}\right) t .
\end{align}
\end{subequations}
Where $v$ is the speed at a time $t$, $x$ is the distance from the origin of the coordinate system,
$a$ is the acceleration and $v_0$ is the initial speed.

We will write the equations in component form along the axes. Let $\vv_0$ be the initial velocity.
In component form these will be
\[
v_{0x} = v_0 \cos \alpha
\]
along the $x$-axis and
\[
v_{0y} = v_0 \sin \alpha
\]
where $v_0 = |\vv_0|$ and $\alpha$ is the angle $\vv_0$ makes with the $x$-axis. Now using \eqref{grav}
and \eqref{v} we may find
\begin{subequations} \label{proj-v}
\begin{align}
v_x &= v_0 \cos \alpha \\
\shortintertext{and}
v_y &= v_0 \sin \alpha - gt . \label{proj-vy}
\end{align}
\end{subequations}

Now, using \eqref{xsuvat} (or by integrating \eqref{proj-v} with respect to $t$), we can obtain the 
standard formulas for the $x$ and $y$ positions during the flight of the projectile:
\begin{subequations}
\begin{align}
x &= (v_0 \cos \alpha) t \label{proj-x} \\
\shortintertext{and}
y &= (v_0 \sin \alpha) t - \frac{1}{2} g t^2 \label{proj-y} .
\end{align}
\end{subequations}

Eliminating $t$ between these equations demonstrates that projectiles follow parabolic
paths, giving
\begin{equation}
y = x \tan \alpha - \frac{g}{2v_0^2 \cos^2 \alpha} x^2
\end{equation}
for the path of the projectile.

Finally we may use these equations to find the maximum height, range and time of flight for a
projectile. The maximum height is obtained when $v_y = 0$ and solving \eqref{proj-vy} with this
condition gives
\begin{equation} \label{proj-height}
t_{H} = \frac{v_0 \sin \alpha}{g} .
\end{equation}
The range is obtained by solving for $y=0$ in \eqref{proj-y} and selecting the non trivial root for $t$
of
\begin{equation}
t_{F} = \frac{2 v_0 \sin \alpha}{g}
\end{equation}
where $t_{F}$ is the time of flight for the projectile. Inserting this into \eqref{proj-x} gives
\[
x = \frac{2 v_0^2 \cos \alpha \sin \alpha}{g}
\]
and recalling that $\sin 2 \alpha = 2 \cos \alpha \sin \alpha$ gives
\begin{equation}
x = \frac{v_0^2 \sin 2 \alpha}{g}
\end{equation}
for the range of the projectile.
\subsection{3D Projectile Motion}

Projectile motion in 3 dimensions works in exactly the same fashion as 2D projectile motion. Here we
will take the $z$-axis to be the vertical and $x$ and $y$ axes to be labelling the surface. The only
component of acceleration is along the $z$-axis, with
\[
a_z = -g
\]
as before. All other equations remain the same.

\section{Basic Aerodynamics}

Of course, the flight of a golf ball is inevitably affected by aerodynamics. As such we need to have
some understanding of fluid mechanics. In particular, we will need to understand how boundary layers
form on and separate from the surface of the golf ball and how this effects the drag on the ball.
First we will review some basic fluid mechanics.

\subsection{Fluid Mechanics}

In this project we will model the air flowing around the ball as being an incompressible fluid. This
is an Eulerian description of fluid flow, viewing the fluid as though the ball is fixed in the centre
of the coordinate system and the fluid moving around the ball \citet{Ruban2014}.

We will not concern ourselves with a full discussion of fluid mechanics from basic principles here, 
instead we will simply state some useful results predominantly following \citet{Ruban2014} and 
\citet{sears}.

Let $\vv$ be the fluid velocity, which is a function of the the position $\rvec$ from the origin of the
coordinate system and of time $t$. Let $\rho$ be the density of the fluid and $p$ the pressure. 
We define the material derivative to be (as a differential operator)
\begin{equation} \nonumber
\frac{D}{Dt} = \frac{\partial}{\partial t} + (\vv \cdot \nabla)
\end{equation}
and represents the rate of change of some quantity within the fluid, while moving with a small element
of the fluid flow. That is, in a description where the fluid moves relative to the coordinate system
the material derivative measures the rate of change as seen by a moving fluid element.

At all points within the fluid the mass continuity equation must apply
\begin{equation} \label{mass-cont}
\matdef{\rho} + \rho \nabla \cdot \vv = 0 .
\end{equation}
This equation encodes the condition that mass is conserved within a fluid without any sources or sinks.
In an incompressible fluid, as we will be primarily conserved with in this project, $\rho$ will not
change with time, and as such $D \rho / D t = 0$. As a consequence, both terms on the left hand side of
\eqref{mass-cont} must be zero everywhere within the fluid, and therefore the equation reduces to
\begin{equation} \label{incompress}
\nabla \cdot \vv = 0
\end{equation}
for an incompressible fluid.

The continuity equation gives one equation for the velocity components $u,v,w$\footnote{
	The convention within fluid mechanics is that the $x,y,z$ velocity components are called $u,v,w$
	respectively. That is \[
		v_x = u, \qquad v_y = v, \qquad v_z = w .
	\]
} within a fluid. In order to specify the pressure and velocity everywhere we therefore require three
more equations to determine the system. These three equations are supplied by considerations of
energy and momentum conservation. Keeping all of these in mind, we may write a
momentum equation in the form
\begin{equation} \label{momentum}
\rho \matdef{\vv} = - \nabla p + \mu \nabla^2 \vv + \ff
\end{equation}
where $\ff$ is body force per unit volume acting on the fluid (for example a gravitational force) 
and $\mu$ is the viscosity of the fluid.

Equations \eqref{incompress} and \eqref{momentum} when taken together form the Navier-Stokes equations
for the velocity and pressure fields within an incompressible fluid. It is well known that these
equations are highly non-linear and exceedingly difficult to solve both analytically and numerically.

The Navier-Stokes equations give rise to a number of fascinating effects within fluid dynamics. In this
project, we are particularly interested in boundary layer effects and turbulence both of which are 
resultant from considerations of the Navier-Stokes equations.

\subsection{Non Dimensional Variables and the Reynolds Number}

\subsection{Boundary Layers}

One of the fundamental ideas within fluid mechanics is that of the no slip condition. The no slip 
condition specifies that when a fluid encounters a solid body it must, at the surface of the body,
take the velocity and temperature of that body. This means that there will be a thin layer of fluid
around the body where the velocity changes from that of the overall stream to match the velocity of
the solid body: this layer is referred to as the boundary layer.

The equations which govern the flow within the boundary layer are a simplified version of the
Navier-Stokes equations, taking into account the order of magnitude of the boundary layer compared to
the size of the body. The derivation of these equations completed by scaling the Navier-Stokes 
based on the assumption that the boundary layer is much smaller than the body size \citet{anderson}.

In 2D these equations look as follows:
\begin{subequations}
\begin{align}
\pd{u}{x} + \pd{v}{y} &= 0 &\text{(Continuity equation)} \\
u\pd{u}{x} + v\pd{u}{y} &= -\frac{1}{\rho} \pd{p}{x} +
\nu \spd{u}{y} &\text{($x$ momentum)} \\
\frac{1}{\rho} \pd{p}{y} &= 0 &\text{($y$ momentum)}
\end{align}
\end{subequations}

\subsection{Boundary Layer Separation and the Magnus Effect}

\subsection{Lift and Drag}

\section{Previous Work on Modelling Golf Ball Flight}