The motion of a golf ball can fundamentally be viewed as a set of forces acting on the ball as it
flies through the air. Before we can obtain a better understanding of the nature of the drag and lift,
or characterise a ball by the lift and drag functions we find,
obtaining a model for the forces acting on the ball during the flight is advisable. 

Here, we follow the paper by \citet{Robinson2013}, which builds a model of the flight based on simple principles as
we desire. This model does neglect some of the subtlety of the fluid dynamics we have discussed,
but is a useful starting point for the analysis.

\section{A Model for Golf Ball Flight}

In \citet{Robinson2013} the authors first discuss the assumptions and limitations of the model at hand.
We will take the reverse to this approach, discussing the features of the model and then mentioning
some of the potential improvements to the model and the drag and lift form they give.

Additionally, the authors give a sample MATLAB script for the setup of the differential equations they
derive and suggest using MATLAB to solve the resultant differential equations of the model numerically. 
This sample code forms the bases of our initial investigations into the trajectories.

\subsection{Lift and Drag}

\citeauthor*{Robinson2013} start by initially discussing the lift and drag forces on a golf ball.
They use the form of the lift and drag in the high Reynolds number limit, given by

\begin{equation} \nonumber
F_{D} = \frac{1}{2} \rho \vv^2 A c_{D}
\end{equation}
and
\begin{equation} \nonumber
F_{L} = \frac{1}{2} \rho \vv^2 A c_{L}
\end{equation}

The use of this form of equation is justified by the authors stating that there is experimental evidence
that golf ball flight always occurs at $Re > 10^3$. We will examine this assumption later.

\subsubsection{Drag}

The authors give a form for $c_D$ based on the spin of the ball
\begin{equation}
c_D = 0.3 + 2.58\times10^{-4}\omega
\end{equation}
where $\omega$ is the modulus of $\vec{\omega}$ the spin vector, in radians per second. This is obtained 
from experimental results found by
\citet{davies1949aerodynamics}. However, the authors elect to take $c_D = 0.45$, rationalising that
this simplification is most suited to the range of spins and speeds which golf balls are likely to
take. The authors also mention the lack of experimental evidence they found to suggest better forms
for $c_D$.

\subsubsection{Lift}

\citeauthor*{Robinson2013} make the following assumptions for the lift on the golf ball:
\begin{itemize}
\item The direction of the lift for is perpendicular to both $\omega$ and $\vv$, that is $\vec{F}_{L} \propto \omega \times \vv$.
\item The lift is not a function of the drag.
\end{itemize}

They they go on to express the lift force as a vector quantity, given by
\begin{equation}
\vec{F}_{L} = \frac{1}{2} \rho A c_{L} V^2 \sin \theta \cdot \hat{n}
\end{equation}
and then using the definition of cross products rewrite this form as
\begin{equation}
\vec{F}_L = \frac{1}{2} \rho A c_L V \cdot \left(\frac{\vec{\omega} \times \vv}{\omega}\right)
\end{equation}
which is the form used in the model.

As with $c_D$, the authors then use experimental data to inform their choice for $c_L$
\subsection{Accounting for the Wind}

\subsection{The Equations of Motion}

% -------------

% Describe the model, provide diagrams, show basic runs.

% Model is a good comprimise between simplicity and flexibility. Can plug in new drag coefficents.
% Some parts of the model are purely heurestic like the form for the lift, others seem just plain wrong.
% The overall structure is good though. Compare basic run to data, show similar shape but incorrect
% carry.

\section{Limitations of the Model}

% Talk about the \citet{Jensen2014Comment} comment, dimensional analysis, lead into talking about why
% we need to improve this model to find a better form for $c_{D}$. Ppotentially a way to estimate the 
% spin ratio. Mention how \citet{Robinson2014Reply} addresses some of the concerns of the comment but does
% not give forms for $c_{D}$. Talk about how the golf ball is in the middle between the high and low
% reynold limits and this will require some matching modelling to find a good form.