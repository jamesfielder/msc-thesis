The motion of a golf ball can fundamentally be viewed as a set of forces acting on the ball as it
flies through the air. Before we can obtain a better understanding of the nature of the drag and lift,
or characterise a ball by the lift and drag functions we find,
obtaining a model for the forces acting on the ball during the flight is advisable. 

Here, we follow the paper by \citet{Robinson2013}, which builds a model of the flight based on simple principles as
we desire. The equations described within \citet{Robinson2013} are not new however: very similar
equations appear in \citet{Bearman1976}. However, the analysis and presentation in \citet{Robinson2013}
is particularly clear, so we will use this as our starting point.

This model does neglect some of the subtlety of the fluid dynamics we have discussed,
but is a useful starting point for the analysis.

\section{A Model for Golf Ball Flight}

In \citet{Robinson2013} the authors first discuss the assumptions and limitations of the model at hand.
We will take the reverse to this approach, discussing the features of the model and then mentioning
some of the potential improvements to the model and the drag and lift form they give.

Additionally, the authors give a sample MATLAB script for the setup of the differential equations they
derive and suggest using MATLAB to solve the resultant differential equations of the model numerically. 
This sample code forms the bases of our initial investigations into the trajectories.

\subsection{Lift and Drag}

\citeauthor*{Robinson2013} start by initially discussing the lift and drag forces on a golf ball.
They use the form of the lift and drag in the high Reynolds number limit, given by

\begin{equation} \nonumber
F_{D} = \frac{1}{2} \rho \vv^2 A c_{D}
\end{equation}
and
\begin{equation} \nonumber
F_{L} = \frac{1}{2} \rho \vv^2 A c_{L}
\end{equation}

The use of this form of equation is justified by the authors stating that there is experimental evidence
that golf ball flight always occurs at $Re > 10^3$. We will examine this assumption later.

\subsubsection{Drag}

The authors give a form for $c_D$ based on the spin of the ball
\begin{equation}
c_D = 0.3 + 2.58\times10^{-4}\omega
\end{equation}
where $\omega$ is the modulus of $\vec{\omega}$ the angular velocity vector, in radians per second. This is obtained 
from experimental results found by
\citet{davies1949aerodynamics}. However, the authors elect to take $c_D = 0.45$, rationalising that
this simplification is most suited to the range of spins and speeds which golf balls are likely to
take. The authors also mention the lack of experimental evidence they found to suggest better forms
for $c_D$.

\subsubsection{Lift}

\citeauthor*{Robinson2013} make the following assumptions for the lift on the golf ball:
\begin{itemize}
\item The direction of the lift for is perpendicular to both $\vec{\omega}$ and $\vv$, that is $\vec{F}_{L} \propto \vec{\omega} \times \vv$.
\item The lift is not a function of the drag.
\end{itemize}

They then go on to express the lift force as a vector quantity, given by
\begin{equation}
\vec{F}_{L} = \frac{1}{2} \rho A c_{L} V^2 \sin \theta \cdot \hat{n}
\end{equation}
and then using the definition of cross products rewrite this as
\begin{equation}
\vec{F}_L = \frac{1}{2} \rho A c_L V \cdot \left(\frac{\vec{\omega} \times \vv}{\omega}\right)
\end{equation}
which is the form used in the model.

As with $c_D$, the authors then use experimental data to inform their choice for $c_L$, once again
following \citet{davies1949aerodynamics} to find that
\begin{equation} \label{cl-rr}
c_L = 3.19\times10^{-1} \left(1 - \exp(-2.48\times10^{-3} \omega)\right)
\end{equation}
fits with a number of different experimental studies of the lift over spheres.

The authors also state that this form completely omits any possibility of a negative Magnus effect 
pushing the ball towards the ground.

\subsection{Accounting for the Wind}

In order to deal with the effect of the wind on the golf ball the authors introduce a new $\vec{v}$
which is defined as being the velocity of the ball relative to the air. That is, if the wind is blowing
with a velocity $\vec{W}$ and the ball moving with velocity $\vv$ with respect to the coordinates then
the ball is moving with velocity $\vec{v} = \vv - \vec{W}$ with respect to the air.

\subsection{The Equations of Motion}

By considering a force balance on the golf ball in flight, the authors obtain
\begin{equation}
m \ddot{\vec{r}} = \underbrace{- \frac{1}{2}\rho A c_D |\vv - \vec{W}|(\vv - \vec{W})}_{1} + \underbrace{\frac{1}{2} \rho A c_L
|\vv - \vec{W}| \left(\frac{\vec{\omega} \times (\vv - \vec{W})}{\omega}\right)}_{2} + \underbrace{m \vec{g}}_{3} .
\end{equation}
Where $\vec{r}$ is the position vector from the origin, taking the $z$-axis to be facing upwards, and $x$ and $y$ in the plane with the ground. The dot here means the time derivative. 
The three terms here are numbered by their origin:
\begin{enumerate}
\item The drag term accounting for the action of the wind.
\item The lift term, written in terms of cross products, accounting for the action of
the wind.
\item The action of gravity on the ball, with $\vec{g} = (0,0,-g)^{T}$.
\end{enumerate}

If we write these into their respective components we obtain, expanding the cross product terms into
the components of vectors
\begin{subequations}
\begin{align}
m \ddot{x} &= -\frac{1}{2} \rho A |\vv - \vec{W}| \left(c_D (V_x - W_x) - c_L \frac{\omega_y (V_z - W_z)
- \omega_z (V_y - W_y)}{\omega} \right) \\
m \ddot{y} &= -\frac{1}{2} \rho A |\vv - \vec{W}| \left(c_D (V_y - W_y) - c_L \frac{\omega_z (V_x - W_x)
- \omega_x (V_z - W_z)}{\omega} \right) \\
m \ddot{z} &= -\frac{1}{2} \rho A |\vv - \vec{W}| \left(c_D (V_z - W_z) - c_L \frac{\omega_x (V_y - W_y)
- \omega_y (V_x - W_x)}{\omega} \right) - mg .
\end{align}
\end{subequations}

In order to solve these equations we rewrite them into a set of 6 first order equations, instead of 
three second order ones, as
\begin{subequations}
\begin{align}
\dot{\vec{r}} &= \vec{v} \\
m \dot{\vec{v}} &= - \frac{1}{2}\rho A c_D |\vv - \vec{W}|(\vv - \vec{W}) + \frac{1}{2} \rho A c_L
|\vv - \vec{W}| \left(\frac{\vec{\omega} \times (\vv - \vec{W})}{\omega}\right) + m \vec{g}
\end{align}
\end{subequations}
This rewriting allows an implementation of the equations which may be integrated using the standard
differential equation solvers within MATLAB.

Solving these equations involves giving 6 initial conditions and 3 parameters: the three coordinates at the start of
the trajectory, the components of the initial velocity, $\rho = 1.22 \text{kg}\,\text{m}^{-3}$ in air,
the diameter of the ball ($D = 42.67$mm\footnote{This is the size of the largest ball allowed by The R\&A,
as was stated in the introduction.}), and the angular velocity vector of the ball. 

We will also assume that $\omega_x = 0$, that is the rotation of the ball is always
``towards'' the $x$-axis and thus only $\omega_y$ and $\omega_z$ are none zero. We will specify these 
components as $\omega_y = |\vec{\omega}| \cos \theta$ and $\omega_z = |\vec{\omega}| \sin \theta$ where $\theta$
is an angle measuring the tilt of the spin axis from the $x$-$y$ plane.

\subsection{An Example Trajectory} \label{rr-example}

In order to show this model working, we will demonstrate solving an example problem in matlab with
values taken from a trajectory provided by The R\&A.

Here, we take the following initial conditions
\[
x = 0, \,\, y = 0, \,\, z = 0, \,\, v_x = 68.221, \,\, v_z = 12.701, \,\, v_y = -4.961, \,\, |\vec{\omega}| = 271.44, \,\, \theta = \frac{169\pi}{180}
\]

In figure \ref{2d-test-rr} we plot the range ($x$-axis) against the height of the the golf ball during
the flight. There are several encouraging features within this plot: the height and range are within
the bounds set by considering the ball as a projectile and completely negating the affect of drag, and
the shape of the trajectory is not simply that of a parabola, implying that the lift and drag are changing how
the ball flies.
\begin{figure}
% This figure is called new-golf right now
\centering
\includegraphics[scale=0.6]{../images/rr_traj_basic.png}
\caption[2D plot of an example Robinson and Robinson model trajectory]{Here we plot the distance of the
golf ball against the height, using the basic \citeauthor*{Robinson2013} model to calculate the trajectory.}
\label{2d-test-rr}
\end{figure}

\begin{figure}
% Also in new-golf
\centering
\includegraphics[scale=0.4]{../images/3d-rr.png}
\caption[3D plot of an example Robinson and Robinson model trajectory]{A 3D plot of the same model as
Figure \ref{2d-test-rr}, showing the deviation due to the spin of the ball in this model.}
\label{3d-test-rr}
\end{figure}

In figure \ref{3d-test-rr} we display a 3 dimensional plot of the trajectory. Here we notice that 
the spin of the ball influences the movement in the $y$ plane fairly considerably, adding a deviation
from a standard projectile which we would not predict by projectile motion alone. This is the action
of the spin on the trajectory.

\section{Limitations of the Model}

Whilst the model given by \citet{Robinson2013} is a good start, it is not without flaws. The authors
themselves admit that there are a number of points where the model could potentially be improved
\begin{itemize}
\item The model only takes into account a positive Magnus effect, but if the separation points were
to move to a different position on the ball this form would no longer apply.
\item Given the earlier analysis in this text, the constant value of $c_D$ is rather troubling.
\item The form of $c_D$ and $c_L$ have no dependency on Reynolds number at all, which contradicts previous
analysis.
\item The spin of the ball is assumed constant throughout the flight. The authors state that in most
situations the loss of spin is low enough to be ignored, citing some evidence that the ball keeps a
significant proportion. This assumption is also presented in \citet{Lieberman2001}, where the simple
differential equation
\[
\dot{\vec{\omega}} = -0.00002 \frac{\vec{\omega} v}{r}
\]
is used to model the spin decay. This constant is so small as to cause almost no change in spin over
the flight, and as such we will not concern ourselves with modelling spin decay.
\end{itemize}

The authors do state, however, that building dependency on speed and spin into their model would be
possible with the analysis they have performed, allowing these limitations to be addressed without 
drastically changing the form of the resultant equations.

The lack of any dependency on speed is addressed within \citet{Jensen2014Comment} by means of dimensional
analysis. Jensen uses a similar analysis to the previous section on drag to determine that if $c_L$ is
independent of the Reynolds number then it cannot, by dimensionality arguments, only be a function of
$\omega$. Instead, in order for the function to have some dependency on spin a possible dimensionless grouping
of $r\omega/\vv$ can be used. Jensen also suggests several other forms which $c_D$ and $c_L$ may take
based solely on dimensional considerations.

\citet{Robinson2014Reply} is the authors of \citet{Robinson2013} reply to the paper by Jensen. Robinson
and Robinson state that the second constant in \eqref{cl-rr}, as found from \citet{davies1949aerodynamics},
is given in units of $1/(\text{rad} \,\text{s}^{-1})$, and as such cancel with the units of $\omega$ and 
overall give an expression which is dimensionless, as we expect $c_L$ to be from our earlier discussion.
The authors go on to discuss the necessity and normality of having a multiplying constant which renders
an expression dimensionally consistent, giving an example based on the spring constant from Hooke's 
Law.

Robinson and Robinson also give a number of other references to experimental data confirming their
assumptions with regards to the dependency of lift on spin, and discuss some other alternative forms
of the lift force with different dependencies on the velocity of the ball.

\subsection{Comparing Trajectories with Data}

In order to improve upon the model by Robinson and Robinson the first thing to do is attempt to validate
the trajectories generated against data provided by The R\&A, and hopefully obtain some understanding
of which model assumptions are incorrect.

If we use the same initial conditions as in section \ref{rr-example} and plot this against a trajectory
provided by The R\&A using their Doppler radar system we obtain Figure \ref{data-rr-2d}. We note that
the model under predicts the height and range of the ball by a considerable degree.

\begin{figure}
% Generated using the code in new golf
\centering
\includegraphics[scale=0.4]{../images/data1-rr-2d.png}
\caption[Plotting the data against an example Robinson and Robinson trajectory]{Here we plot the Robinson
and Robinson model in blue against data provided by The R\&A in a dashed line. The significant change in maximum height
and carry of the ball highlight the fact that there are significant improvements to be made to the model.}
\label{data-rr-2d}
\end{figure}

In Figure \ref{data-rr-3d} a 3D plot of this data is shown. Here we see that the model initially shows
good agreement with the data but as the ball approaches the top of the flight the model and the data
deviate considerably from each other, with the actual trajectory curving back towards the $y=0$ line.
This effect is completely absent from the curve given by the Robinson and Robinson model and suggests
that the drag or lift need significant changes in order to fit with the data.

\begin{figure}
% Generated using the code in new golf
\centering
\includegraphics[scale=0.5]{../images/data1-rr-3d.png}
\caption[Plotting the data against an example Robinson and Robinson trajectory in 3D]{Here we plot the Robinson
and Robinson model in blue against data provided by The R\&A plotted with blue circles. While the model initially fits
the data well, as the ball approaches the top of the flight the two curves deviate significantly.}
\label{data-rr-3d}
\end{figure}

One hint as to the failings of this model is to plot the speed of the projectile against
the distance along the $x$-axis, as in Figure \ref{data-rr-v}. We see that while the curves start
together, they immediately deviate from each other. By $x=50$ the Robinson and Robinson model predicts 
the speed to be $10\text{m}\,\text{s}^{-1}$ less than the speed obtained from the data.
\begin{figure}
% Also in new golf
\centering
\includegraphics[scale=0.56]{../images/data1-rr-v.png}
\caption[Plot of speed in Robinson and Robinson model against the speed in the data]{Here we plot the
speed of the projectile in solid line against the speed as measured by the radar in dashed line. Note that while
the curves start with the same initial condition they quickly deviate from each other.}
\label{data-rr-v}
\end{figure}

On calculating the Reynolds number for the flight, at the start of the trajectory the Reynolds number
is approximately $1.6\times10^5$, which is just where we believe the supercritical regime begins. This
would invalidate the assumption which Robinson and Robinson make that the drag is likely to be nearly
constant through the entire motion of the projectile. Instead, the drag on the ball must initially be
very different to the value which Robinson and Robinson assumes it to be throughout the flight. This
difference accounts for the variation of the speed of the ball immediately after being hit, as the 
drag on the real ball is much less than the model drag, slowing the ball less and keeping
the ball in the critical or supercritical Reynolds numbers for longer.

% Talk about the \citet{Jensen2014Comment} comment, dimensional analysis, lead into talking about why
% we need to improve this model to find a better form for $c_{D}$. Ppotentially a way to estimate the 
% spin ratio. Mention how \citet{Robinson2014Reply} addresses some of the concerns of the comment but does
% not give forms for $c_{D}$. Talk about how the golf ball is in the middle between the high and low
% reynold limits and this will require some matching modelling to find a good form.

\section{Summary and Improvements}

\citet{Robinson2013} give fairly well justified reasons to take the lift to be in the form they suggest
including referencing a number of experimental results that fit well with the form they give. Thus,
we will focus our investigations primarily on finding a better form for the drag function on the ball,
as this will likely yield the biggest improvements to the model.

The assumption that the ball never reaches the supercritical Reynolds numbers appears, by investigation
of the data above (and from other data sets obtained from The R\&A) seems to be false, so finding a
functional form of $c_D$ which takes into account the drag crisis for Reynolds numbers within the
range encountered during golf ball flight is a good starting point for investigations. Indeed, within
\citet{Lieberman2001} the $c_D$ is also taken to be dependent on Reynolds number and other
dimensionless parameters related to the spin. However, we will focus simply on dependency on Reynolds
number.

Overall, the model given in \citet{Robinson2013} seems to be a good starting point for understanding
the forces on the ball and finding trajectories, being similar in nature to the results found in previous
works such as \citet{Bearman1976}.